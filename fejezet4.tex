%----------------------------------------------------------------------------
\chapter*{Továbbfejlesztési lehetőségek és következtetések} \label{fejezet4}
%----------------------------------------------------------------------------
\onehalfspacing
A Cronotus projekt jelenlegi állapotában tartalmaz minden olyan alap funkciót, mely szükséges az alapszintű működéséhez
és a cél eléréséhez, viszont számos továbbfejlesztési lehetőség rejlik benne, melyekkel a felhasználói élményt tovább
lehetne javítani, illetve a funkcionalitást bővíteni:

\begin{itemize}
	\item \textbf{Email alapú értesítések:} Email alapú értesítések bevezetése, melyek a felhasználókat értesítenék az eseményekről, melyek érdeklik őket, vagy azokról, melyekben részt vesznek.
	
	\item \textbf{Google térkép integráció:} A Google térkép integrációja, mely lehetővé tenné a felhasználók számára, hogy megtekinthessék az események helyszínét a térképen.
	
	\item \textbf{Visszajelzés rendszer bevezetése:} Egy visszajelzés rendszer bevezetése, mely lehetővé tenné a felhasználók számára, hogy értékeljék az eseményeket, illetve a szervezőket, és ezáltal segítsék a többi felhasználót a döntésben.
	
	\item \textbf{Barátok:} Egy rendszer, ami lehető teszi azt, hogy felhasználókat jelölhessünk be barátként, és így könnyebben megtalálhassuk azokat az eseményeket, melyek érdekelhetik őket.
	
	\item \textbf{Statisztikai kimutatások:} Időszakos statisztikai kimuatatások, amelyek jelzik, hogy egy adott időintervallumban mely sportokat végzett egy felhasználó aktívan, illetve milyen eseményeken vett részt.
\end{itemize}

