%----------------------------------------------------------------------------
\chapter{A projekt menedzselése} 
%----------------------------------------------------------------------------

Annak érdekében, hogy a Cronotus projekt fejlesztési fázisa célra vezető és zökkenőmentes legyen, hasznos különféle projektmenedzselési
eszközöket felhasználni.

\section{Verziókezelés Git segítségével}

A Cronotus projekt verziókezelése Git\cite{gitdocs} segítségével történt. A Git azért volt egy hasznos eszköz a projekt keretein belül,
mivel így könnyen számon lehet tartani az egyes verziók közötti különbséget, illetve ha az újonnan beiktatott kód nem megfelelő, könnyen el lehet
érni korábbi, működő verziókat. A projekt verziókezelésére külön git branch-ek voltak használva, így minden nagyobb, elkülöníthető
funkció egy külön ágat kapott a jobb átláthatóság érdekében.

A forráskód publikusan elérhető a \url{https://www.github.com/cronotus} weblapon.

Itt a projekt kliens oldali kódja megtalálható a \url{https://www.github.com/cronotus/frontend}, míg a szerver oldali kódja a \url{https://www.github.com/cronotus/backend} címen érhető el.