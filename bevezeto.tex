%----------------------------------------------------------------------------
\chapter{Bevezető}%\addcontentsline{toc}{chapter}{Bevezető}
%----------------------------------------------------------------------------

Annak ellenére, hogy az információ közlése és elérése napjainkban rendkívül gyors és olcsó, mégis előfordul, hogy alapvető tevékenységeket igen csak maradi módon kezelnek az emberek. Többek között idetartozik egyes sportesemények koordinálása is. A résztvevők gyakran hajlamosak egy nem erre a célra készített platformra hagyatkozni, ami természetesen nem mindig működik úgy, ahogy ideális lenne. Ennek eredményeként megtörténik, hogy lemaradnak az emberek olyan információkról, ami egyébként releváns lett volna számukra.

A Cronotus projekt ennek a problémának a megoldását egy ingyenes és könnyen kezelhető platfrom biztosításával közelíti meg, ami arra törekszik, hogy kiküszöbölje az esetelges információk elvesztését, vagy akár figyelmen kívül hagyását.

A dolgozat elsősorban bemutatja a projekt általános működését, párhuzamba helyezi más ehhez hasonló internetes platformokkal, szemünk elé tárja az esetleges előnyöket és hátrányokat más szolgáltatásokkal szemben.

A továbbiakban a dolgozat ismerteti velünk mélyebben a Cronotus szerkezetét, ahol felépítés szerinti rétegekre bontja, majd ezeket részletekbe bocsátkozva tárgyalja. Bemutatja azt, hogy hogyan működik a web kliens-t kiszolgáló szerver, milyen biztonsági intézkedéseket biztosít, illetve hogyan kezeli a felmerülő hibákat. Betekintést biztosít abba, hogy hogyan érkezik el a kívánt információ az adatbázistól egészen a felhasználó felületig.

A következő fejezetek segítségével szeretném bemutatni a felhasznált eszközöket és módszereket, melyek elősegítik a Crontous helyes működését. Végezetül egy következtetést vonok le, illetőleg továbbfejlesztési lehetőségeket tárok az olvasó elé, melyek fényt derítenek a projekt jövőbeli élettartamára.

A paltform készítése 2023 októberében kezdődött a Codespring szoftverfejlesztő cég koordinálásával. Megemlítendő, hogy a felhasznált technológiák visszatükrözik azt a tudást, amit a Marosvásárhelyi Codespring Mentorprogram keretein belül sajátíthattam el, ami 2022 őszi időszakában vette kezdetét. A projektet a koordináló cég segítségével egyedül fejlesztettem.

Az alkalmazás neve, azaz a Cronotus, egy összetett lingvisztikai ihletből merített szó. A név első fele a görög mitológiai Krónosztól származik, akit az idő isteneként ismerünk, míg a név második fele a ``mozgás'' szó latin fordításából, a 'motus' szóból ered.