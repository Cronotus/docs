%----------------------------------------------------------------------------
\chapter*{Összefoglaló}\addcontentsline{toc}{chapter}{Összefoglaló}
%----------------------------------------------------------------------------

Dolgozatomban differenciálegyenletek megoldásával foglalkoztam, amelyet különböző programozási technológiák segítségével valósítottam meg. Először ismertettem a differenciálegyenletek numerikus megoldásának elméleti alapjait, majd megvizsgáltunk és levezettünk három numerikus módszert az Euler-, a Runge-Kutta és a Dormand-Prince módszereket. Ezek közül a mai technológiákban leginkább használatos Dormand-Prince algoritmus esetében megnéztük, hogy milyen szoftverekben tálálhatjuk meg, mint alapértelmezett differenciálegyenlet megoldó. A továbbiakban ismertettem két modellt, a leukémia betegség alap modelljét és a hullámmozgás modelljét, ezzel is kihangsúlyozva a téma fontosságát, hogy mennyire fontos az időtényező bizonyos problémák egyenleteinek medoldásánál. Ezek után részletesen is mégnéztük, hogy milyen szoftvereket alkalmaztam és alkottam a differenciálegyenletek és rendszerek megoldására. A szoftvereket két kategóriába osztotottuk fel, az első a már létező szoftverek kategóriája, a másik pedig az általam megvalósított szoftverek csoportja volt. Az első kategóriában ismertettem két technológiát, a Matlab által nyúltott ode45 beépített megoldót és a Boost könyvtárcsomagban található Odeint nevű könyvtárat. Az általam írt szofverek csoportjában négy megvalósítást mutattam be ezek a Matlab, Java, C++ és Android technológiák segítségével készültek. Emellett megnéztük, hogy mennyire hatékonyan lehet párhuzamosítani a differenciálegyenletek megoldását CUDA technológia segítségével és a grafikus kártyát (GPU-t) felhasználva. Végül kiértékeltük a tesztelés során kapott eredményeket és összehasonlítottuk a különböző programokat, kiemelve azok erősségeit és gyengéit. Majd levontuk a következtetéseket, hogy melyik technológia irányában érdemes tovább haladni és melyik az, amelyikkel nem éri meg foglalkozni.

Jövőbeli terveimet illetően szeretnék jobban elmerülni a GPU-n történő differenciálegyenletek megoldásának módszereiben, valamint ezek alkalmazását kipróbálni és tanulmányozni a parciális differenciálegyenletek területén (PDE). Továbbá érdemesnek tartom a C++ szoftver továbbfejlesztését és egy differenciálegyenlet megoldó könyvtár megalkotását, amely ingyenesen használható és nyílt forráskódú lenne.